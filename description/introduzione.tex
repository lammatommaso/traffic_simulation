\documentclass[main.tex]{subfiles}

\begin{document}

\subsection{Scopo del progetto}

Il progetto è finalizzato alla simulazione del traffico su una città modellizzata come un grafo diretto, ovvero con doppi sensi e sensi unici.
La domanda che ci siamo posti all' inizio della programmazione era se in un qualche modo i sensi unici potessero essere vantaggiosi in alcune condizioni
rispetto ai doppi sensi a parità di capienza, perché semplicemente sostituire due corsie alternate con una sola corsia diminuirebbe la superficie della città
e a parità di numero di automobili aumenterebbe la densità e porterebbe sicuramente ad uno svantaggio.

\subsection{Modello}

La città è rappresentata come un grafo a griglia, e tramite una probabilità impostata dall'utente è possibile controllare la frazione di sensi unici.
All'inizio della simulazione la matrice di adiacenza è creata e tale probabilità è la probabilità che in fase di creazione un senso alternato diventi senso unico.

Le strade sono modellizzate come dei contenitori di automobili che sanno soltanto il numero di automobili che contengono, però dato che ogni automobile conosce il suo offset sulla strada
questo modello è analogo ad una strada visibile come un array in ogni casella del quale è possibile mettere al più un numero di macchine pari al numero di corsie della strada.

In questa simulazione il traffico emerge da un controllo sul movimento delle automobili che impedisce
alla singola automobile di entrare in una casella che contenga un numero di macchine pari al numero di corsie della strada.

Ogni automobile conta il numero di volte in cui passa da una casella a quella successiva e anche il numero di volte in cui
è costretta a fermarsi a causa del traffico.

\subsection{Modalità e parametri della simulazione }

L'applicazione permette di fare due tipi di simulazione, una grafica e una "batch" (non grafica); in quella batch si può scegliere tra due simulazioni, "car increment" e bla bla bla

% \subsubsection{One way increment}

% \subsubsection{Car increment}

% \subsubsection{Modalità grafica}
% La modalità grafica permette di costruire una simulazione scegliendo tra diversi parametri; questi sono:
% \begin{itemize}
%     \item \textbf{Dimensione} del grafo (numero righe, numero colonne)
%     \item \textbf{Lunghezza} delle strade (minima e massima)
%     \item \textbf{Percentuale di sensi unici}, tra 0 (nessun senso unico) e 1 (solo sensi unici)
%     \item \textbf{Numero di corsie} per strada
%     \item \textbf{$\sigma$, media} gaussiana %TODO
% \end{itemize}

Dopo aver scelto questi parametri, l'utente vedrà un grafo dove potrà selezionare quali sono i nodi di partenza e di arrivo, dove si genereranno e cercheranno di arrivare le macchine. Sono disponibili alcuni preset, ma l'utente può selezionare i nodi a piacere. 

Infine, quando si avvierà effettivamente la simulazione, l'utente vedrà colorarsi gli archi - le strade - a seconda della loro occupazione:
\begin{itemize}
    \item \textbf{Strada blu}: la strada è vuota o contiene una macchina
    \item \textbf{Strada verde}: la strada è occupata al massimo al 50\% della capacità
    \item \textbf{Strada arancione}: l'occupazione della strada è tra il 50\% e il 75\% della capacità 
    \item \textbf{Strada rossa}: la strada è occupata per più del 75\%.
\end{itemize}

La scelta di non rappresentare direttamente le strade vuote è stata determinata dal meccanismo con cui le strade sono colorate: infatti, il server restituisce al "client", la parte di applicazione responsabile della visualizzazione grafica, le strade che hanno almeno una macchina; il client controlla poi il rapporto tra l'occupazione della strada e la dimensione della stessa, e a seconda del valore colora gli archi-strade. Questo permette al client di ricevere solo le strade che hanno \textit{almeno} una macchina, e non \textbf{tutte} le strade; infatti, la cardinalità delle strade che sono occupate è decisamente inferiore alla cardinalità dell'insieme di tutti archi del grafo, e questo permette un aumento delle prestazioni - in quanto bisogna controllare meno archi - al costo di una perdita minima di precisione, in quanto le strade blu possono essere vuote o contenere una sola macchina. 

Le strade blu quindi rappresentano sostanzialmente quelle strade che sono state occupate dalle macchine, per permette all'utente di visualizzare i percorsi compiuti da queste. 

\end{document}
