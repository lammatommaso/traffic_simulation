\documentclass[main.tex]{subfiles}
\begin{document}
\subsection{Analisi città esistenti}
È possibile, oltre che interessante, effettuare simulazioni sul traffico su città esistenti: 

step 1. ottenere una rappresentazione a grafico della città; si può fare con geojson e il progetto openstreetmap, oppure https://www.cityjson.org/datasets/, oppure [cerca su google roba a caso] oppure [progetto che non mi ricordo]

step 1.1: conversione geojson -> gexf http://ignacioarnaldo.github.io/OpenStreetMap2Graph/
sorgenti progetto: https://github.com/ignacioarnaldo/OpenStreetMap2Graph

step 2. creare la matrice di adiacenza compatibile col nostro programma (molto facile farlo da gexf)

step 3. visualizzare il grafico: gexf 
\\
\\
GeoJSON è un formato per la memorizzazione di geometrie spaziali, nel quale abbiamo gli attributi descritti attraverso la JavaScript object notation.
Possiamo rappresentare punti, spezzate, geometrie e collezioni che le contengono.
\\
\\
\textbf{Esempio:}
\begin{lstlisting}
{
  "type": "FeatureCollection",
  "features": [
    {
      "type": "Feature",
      "geometry": {
        "type": "Point",
        "coordinates": [11.1215698,46.0677293]
      },
      "properties": {
        "name": "Fontana dell'Aquila"
      }
    },
    {
      "type": "Feature",
      "geometry": {
        "type": "LineString",
        "coordinates": [
           [11.1214686,46.0677385],[11.121466,46.0677511],
           [11.1213806,46.0681452],[11.1213548,46.0682642],
           [11.1213115,46.0684385],[11.1212897,46.0685261],
           [11.1212678,46.0686443]
        ]
      },
      "properties": {
        "lanes": 1,
        "name": "Via Rodolfo Belenzani"
      }
    }
  ]
}
\end{lstlisting}
Si rivela quindi un buon formato per rappresentare il grafo di una città tramite archi e nodi.
Inoltre grazie alla sua flessibilità è possibile trovare online città reali, costruite in questo formato.
geojson, formato strano usato da libreria javascript 
\end{document}