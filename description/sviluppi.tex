\documentclass[main.tex]{subfiles}
\begin{document}
\subsection{Analisi città esistenti}
È possibile, oltre che interessante, effettuare simulazioni sul traffico su città esistenti: 

step 1. ottenere una rappresentazione a grafico della città; si può fare con geojson e il progetto openstreetmap, oppure https://www.cityjson.org/datasets/, oppure [cerca su google roba a caso] oppure [progetto che non mi ricordo]

step 1.1: conversione geojson -> gexf http://ignacioarnaldo.github.io/OpenStreetMap2Graph/
sorgenti progetto: https://github.com/ignacioarnaldo/OpenStreetMap2Graph

step 2. creare la matrice di adiacenza compatibile col nostro programma (molto facile farlo da gexf)

step 3. visualizzare il grafico: gexf 
\\
\\
GeoJSON è un formato per la memorizzazione di geometrie spaziali, nel quale abbiamo gli attributi descritti attraverso la JavaScript object notation.
Possiamo rappresentare punti, spezzate, geometrie e collezioni che le contengono.
\\
\\
\textbf{Esempio:}
\begin{lstlisting}
{
  "type": "FeatureCollection",
  "features": [
    {
      "type": "Feature",
      "geometry": {
        "type": "Point",
        "coordinates": [11.1215698,46.0677293]
      },
      "properties": {
        "name": "Fontana dell'Aquila"
      }
    },
    {
      "type": "Feature",
      "geometry": {
        "type": "LineString",
        "coordinates": [
           [11.1214686,46.0677385],[11.121466,46.0677511],
           [11.1213806,46.0681452],[11.1213548,46.0682642],
           [11.1213115,46.0684385],[11.1212897,46.0685261],
           [11.1212678,46.0686443]
        ]
      },
      "properties": {
        "lanes": 1,
        "name": "Via Rodolfo Belenzani"
      }
    }
  ]
}
\end{lstlisting}
Si rivela quindi un buon formato per rappresentare il grafo di una città tramite archi e nodi.
Inoltre grazie alla sua flessibilità è possibile trovare online città reali, costruite in questo formato.
geojson, formato strano usato da libreria javascript 

\subsection{Topological Data Analysis}
Una recente branca della matematica, chiamata \ii{topological data analysis}, permette di confrontare dati di traffico da un punto di vista topologico.
\`{E} infatti possibile a partire dalla funzione numero di macchine medio, definita sulle strade(ovvero il grafo), costruire un diagramma di persistenza o codice a 
barre che rappresenta fedelmente il dato. La distanza tra due codici a barre, più semplice da calcolare della distanza nello spazio delle funzioni definite sul grafo,
permette di stimare la distanza effettiva tra i dati di traffico. Tali diagrammi di persistenza o codici a barre sono costruiti vedendo come variano le dimensioni dei gruppi di omologia
degli insiemi di livello della funzione numero di auto. Nel caso di un grafo gli unici gruppi di omologia non banali sono il gruppo delle componenti connesse e quello dei cicli indipendenti.
In futuro si potrebbe pensare di implementare un codice che produca questi diagrammi di persistenza e vedere come i sensi unci interagiscono con la topologia dei dati di traffico.
Per un esempio di applicazione della TDA al traffico si veda \cite{barcodes}.

\subsection{Semafori e Rotonde}
Al momento agli incroci il traffico generato è dovuto soltanto alla congestione e non a rotonde o semafori.
In futuro si potrebbero aggiungere dei semafori o delle rotonde, per rendere le città più realistiche, ma tale aggiunta
sicuramente andrebbe a coprire ulteriormente gli effetti dei sensi unici.
Una cosa interessante potrebbe essere in alternativa costruire una città con soli semafori, e al variare dei sensi unici,
tramite un algoritmo di geometric deep learning imparare la funzione che a ogni nodo associa un duty cycle(o due per le due direzioni) in modo 
da ridurre l'indice di traffico. In seguito si potrebbero confrontare le diverse funzioni di duty cycle al variare del numero di sensi unici in modo statistico
o ricorrendo alla TDA.
La formalizzazione del geometric deep learning è molto recente ( Maggio 2021), per avere un' idea di come funziona vedere \cite{2021geo}.

\end{document}