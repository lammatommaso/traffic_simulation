\documentclass[main.tex]{subfiles}
\begin{document}
    Nella simulazione che incrementa i sensi unici a tre corsie ci aspettavamo un minimo dell'indice di traffico per una frazione di sensi unici diversa da 0.0.
    Ciò non è emerso dalle simulazioni e abbiamo concluso che ci sono tre possibilità\footnote{Esclusi eventuali bug nel codice.}:
    \begin{itemize}
        \item Effettivamente non è vantaggioso sostituire a sensi alternati sensi unici, anche se a tre corsie.
        \item I parametri della simulazione sono troppi e stiamo simulando fuori dal range in cui questo effetto è visibile.
        \item \`{E} molto più rilevante il modo di disporre i sensi unici che la loro frazione.
    \end{itemize}

    La terza di queste possibilità è giustificata dal fatto che, come si nota in Fig. \ref{fig:8}, la dispersione dell'indice di traffico cresce all'aumentare dei
    sensi unici. Quindi data un'elevata frazione di sensi unici il modo di disporli potrebbe essere così influente da coprire gli effetti della nostra simulazione,
    dato che nella nostra simulazione non è possibile controllare la disposizione di tali sensi unici.
    Se il problema è questo, una possibile soluzione futura potrebbe essere data dalla \ii{topological data analysis}.
\end{document}