\documentclass[main.tex]{subfiles}
\begin{document}
    \subsection{Classi}
        \paragraph{car.cpp}
            La classe \bb{Car} ha come attributi privati:
            \begin{itemize}
                \item \bb{short int \_steps}, ovvero il numero di volte in cui l'automobile si è spostata effettivamente nel reticolo.
                \item \bb{short int \_stops}, ovvero il numero di volte in cui l'automobile ha dovuto fermasi a causa del traffico.
                \item \bb{short int \_offset}, ovvero la posizione all'interno della strada.
                \item \bb{bool \_at\_destination}, che diventa vera quando l'auto giunge a destinazione.
                \item \bb{short int \_delay}, ovvero un ritardo nella partenza, poiché dato che tutte le auto
                    sono generate con offset nullo, per evitare sovrapposizioni devono partire una alla volta.
            \end{itemize}
        \paragraph{road.cpp}
            La classe \bb{Road} ha come attributi privati:
            \begin{itemize}
                \item \bb{short int \_car\_number}, ovvero il numero di auto presenti nella strada.
                \item \bb{short int \_road\_length}, ovvero la lunghezza della strada.
                \item \bb{short int \_width}, ovvero il numero di corsie della strada, uguale a 1 nelle strade che fanno parte di un doppio senso,
                    e variabile per i sensi unici.
                \item \bb{static *}, dove * si riferisce ai parametri statistici delle strade spiegati nella sezione precedente, messi come
                    \ii{static} per migliorare l'uso della memoria.
            \end{itemize}
        \paragraph{node.cpp}
            La classe \bb{Node} ha come unico attributo privato:
            \begin{itemize}
                \item \bb{short int \_index}, ovvero l'etichetta del nodo.
            \end{itemize}
        \paragraph{city.cpp}
            La classe \bb{City} ha come attributi privati:
            \begin{itemize}
                \item \bb{Node** \_path} e \bb{short int** \_distance}, che servono a determinare il percorso più efficiente tra due nodi.
                \item \bb{void \_floyd\_warshall}, che dalla matrice di adiacenza determina i due attributi precedenti.
                \item \bb{Node* \_node\_set}, ovvero l'insieme dei nodi della città.
                \item \bb{Road** \_adj\_matrix}, ovvero la matrice di adiacenza del grafo diretto che rappresenta la città.
                \item \bb{short int \_n\_rows} e \bb{short int \_n\_coloumns}, che sono ripettivamente il numero di righe e colonne della griglia che è la città.
                \item \bb{float \_oneway\_fraction}, ovvero un parametro che in generazione determina la probabilità di avere sensi unici,
                    tale parametro può essere diverso dall'effettiva frazione, soprattutto per città piccole, ma nei grafici riportiamo sempre la 
                    frazione effettiva invece che questo parametro.
            \end{itemize}
        \paragraph{simulator.cpp}
            La classe \bb{Simulator} ha necessita di due \ii{struct} aggiuntive:
            \begin{enumerate}
                \item \bb{Car\_Info}, che ha come attributi il percorso di un'auto, il nodo appena passato, e un puntatore a macchina per gestire i dati di traffico.
                \item \bb{Result}, che contiene le statistiche rilevanti della simulazione.
            \end{enumerate}
            Inoltre ha come attributi privati:
            \begin{itemize}
                \item \bb{std::vector$\langle$Car\_Info$\rangle$ \_car\_vector}, tale vettore è usato per associare ad ogni auto un percorso ed un nodo tramite l'indice 
                    all'interno del std::vector. Tale implementazione è necessaria in quanto a ogni iterazione viene chiamato un \ii{sort} che permette di 
                    muovere prima le auto con offset maggiore, riducendo il numero di iterazioni necessarie.
                \item \bb{Result \_result}, ovvero i risulati della simulazione.
                \item \bb{short int \_cars\_at\_destination}, fa proseguire la simulazione finché non eguaglia il numero di automobili generate.
                \item \bb{City \_city}, ovvero la città su cui viene effettuata la simulazione.
                \item \bb{short int \_car\_number}, ovvero il numero di auto generate.
                \item \bb{void \_mv\_car(int car\_index)}, che gestisce tutti i controlli del traffico e muove le automobili.
                    Tale metodo prende la posizione della macchina in \_car\_vector, poiché senza di esso non sappiamo nulla della macchina.
                \item other useful inline functions.
            \end{itemize}
        \paragraph{js\_interface.cpp} 
            La classe \ii{js\_interface.cpp} serve alla grafica, scritta in javascript per interagire con il codice della simulazione in C++.
        \paragraph{numpy\_parser.cpp}
        La classe \ii{numpy\_parser.cpp} serve a creare file python con numpy.array leggibili dalla libreria di graficamento \ii{matplotlib}, 
            a partire da std::vector del C++.
    
    \subsection{Considerazioni}
        La scelta di avere oggetti che essenzialmente sono solo uno o due interi (short) è dovuta ad alcuni problemi nella creazione della matrice di adiacenza.
        Infatti usare strutture più ingombranti causava errori di segmentazione durante la creazione della città, probabilmente dovuto al fatto che la memoria
        occupata dalla matrice di adiacenza eccedeva quella riservata al programma.
        La scelta di avere classi non interagenti che possono interagire solo tramite la classe \ii{Simulator}, è dovuta a problemi di ricorsività
        emersi durante le prime versioni del programma. 
    
\end{document}